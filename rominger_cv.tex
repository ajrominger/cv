%%%%%%%%%%%%%%%%%%%%%%%%%%%% Document Setup %%%%%%%%%%%%%%%%%%%%%%%%%%%%

\documentclass[10pt]{article}

% This is a helpful package that puts math inside length specifications
\usepackage{calc}

% Layout: Puts the section titles on left side of page
\reversemarginpar

%% Set up for letter-sized paper
\usepackage[paper=letterpaper,
marginparwidth=0.95In,     % Length of section titles
marginparsep=0in,       % Space between titles and text
margin=1in,               % 1 inch margins
includemp]{geometry}

%% More layout: Get rid of indenting throughout entire document
\setlength{\parindent}{0in}

%% Reference the last page in the page number
\usepackage{fancyhdr,lastpage}
\pagestyle{fancy}
\fancyhf{}\renewcommand{\headrulewidth}{0pt}
\fancyfootoffset{\marginparsep+\marginparwidth}
\newlength{\footpageshift}
\setlength{\footpageshift}
{0.5\textwidth+0.5\marginparsep+0.5\marginparwidth-2in}
\lfoot{\hspace{\footpageshift}%
  \parbox{4in}{\, \hfill %
    \arabic{page} of \protect\pageref*{LastPage}
    \hfill \,}}

%% PDF bookmarks
\usepackage{color,hyperref}
\definecolor{darkblue}{rgb}{0.0,0.0,0.3}
\hypersetup{colorlinks,breaklinks,
            linkcolor=darkblue,urlcolor=darkblue,
            anchorcolor=darkblue,citecolor=darkblue}
% for URLs
\usepackage{verbatim}

%%%%%%%%%%%%%%%%%%%%%%%% End Document Setup %%%%%%%%%%%%%%%%%%%%%%%%%%%%


%%%%%%%%%%%%%%%%%%%%%%%%%%% Helper Commands %%%%%%%%%%%%%%%%%%%%%%%%%%%%

% The title (name) with a horizontal rule under it
%
% Usage: \makeheading{name}
%
% Place at top of document. It should be the first thing.

\newcommand{\makeheading}[1]{
  \hspace*{-\marginparsep minus \marginparwidth}%
  \begin{minipage}[t]{\textwidth+\marginparwidth+\marginparsep}%
    \begin{center}
      {\large \bfseries #1}\\[-0.15\baselineskip]%      
      \rule{\columnwidth}{1pt}%
    \end{center}
  \end{minipage}
}

% The section headings
%
% Usage: \section{section name}
%
% Follow this section IMMEDIATELY with the first line of the section
% text. Do not put whitespace in between. That is, do this:
%
%       \section{My Information}
%       Here is my information.
%
% and NOT this:
%
%       \section{My Information}
%
%       Here is my information.
%
% Otherwise the top of the section header will not line up with the top
% of the section. Of course, using a single comment character (%) on
% empty lines allows for the function of the first example with the
% readability of the second example.
\renewcommand{\section}[2]%
        {\pagebreak[3]\vspace{1.3\baselineskip}%
         \phantomsection\addcontentsline{toc}{section}{#1}%
         \hspace{0in}%
         \marginpar{
         \raggedright \scshape #1}#2}

% tabular styles for including dates in list of CV content
\usepackage{array, xcolor}
\definecolor{lightgray}{gray}{0.8}
% \newcolumntype{L}{>{\raggedleft}p{0.19\textwidth}}
\newcolumntype{L}{p{0.19\textwidth}}
\newcolumntype{R}{p{0.81\textwidth}}
\newcommand\VRule{\color{lightgray}\vrule width 0.5pt}

% To compress itemized environments
\usepackage{enumitem}
\setlist{nolistsep}
\setlist{noitemsep}


% To add some paragraph space between lines.
% This also tells LaTeX to preferably break a page on one of these gaps
% if there is a needed pagebreak nearby.
\newcommand{\blankline}{\quad\pagebreak[3]}
\newcommand{\halfblankline}{\quad\vspace{-0.5\baselineskip}\pagebreak[3]}


%
% NOTE: \rcollength is the width of the right column of the table
%
\newlength{\rcollength}\setlength{\rcollength}{2.75in}%


% THIS IS THE PRIMARY COMMAND USED FOR ENTERING CV DATA!!!!
\newcommand{\cventry}[2]{%
\begin{tabular}[t]{R !{\VRule} L}
  % \parbox[t][][t]{0.81\textwidth}{#2}
  \begin{minipage}[t][][t]{0.81\textwidth}
    #2
  \end{minipage}
  &
  \parbox[t][][t]{0.19\textwidth}{#1}
\end{tabular}%
\vspace{0.25em}
}

% \setlength{\itemsep}{0em}


%% highlights a name in bibliography
% \newcommand{\makeauthorbold}[1]{%
% \DeclareNameFormat{author}{%
%   \edef\tempname{{#1}}%
%   \ifnumequal{\value{listcount}}{1}
%     {\ifnumequal{\value{liststop}}{1}
%       {\expandafter\ifstrequal\tempname{##1}{\textbf{##1\addcomma\addspace ##4\addcomma\isdot}}{##1\addcomma\addspace ##4\addcomma\isdot}}
%       {\expandafter\ifstrequal\tempname{##1}{\textbf{##1\addcomma\addspace ##4}}{##1\addcomma\addspace ##4}}}
%     {\ifnumless{\value{listcount}}{\value{liststop}}
%       {\expandafter\ifstrequal\tempname{##1}{\textbf{\addcomma\addspace ##1\addcomma\addspace ##4}}{\addcomma\addspace ##1\addcomma\addspace ##4}}
%       {\expandafter\ifstrequal\tempname{##1}{\textbf{\addcomma\addspace ##1\addcomma\addspace ##4\addcomma\isdot}}{\addcomma\addspace ##1\addcomma\addspace ##4\addcomma\isdot}}%
%     }%
% }%
% }



%%%%%%%%%%%%%%%%%%%%%%%% End Helper Commands %%%%%%%%%%%%%%%%%%%%%%%%%%%

%%%%%%%%%%%%%%%%%%%%%%%%% Begin CV Document %%%%%%%%%%%%%%%%%%%%%%%%%%%%

\begin{document}
% \makeauthorbold{Rominger}
\makeheading{Andrew J.~Rominger}

\section{Contact Information}
%
\begin{tabular}[t]{p{\textwidth-\rcollength}p{\rcollength}}
  \multicolumn{2}{l}{Santa Fe Institute} \\  
  1399 Hyde Park Road &\textit{E-mail:} \verb|rominger@santafe.edu|\\
  Santa Fe, New Mexico 87501 USA & \textit{Web:} \verb|nature.berkeley.edu/~rominger|\\
\end{tabular}

\section{Education}
%
\cventry{2016}{
  {\bf University of California, Berkeley} \\
  Ph.D. Environmental Science, Policy \& Management \\
  {\it Committee:} Rosemary Gillespie, John Harte \& Charles Marshall \\
  {\it Dissertation:} The statistical mechanics of biodiversity
}
%
\cventry{2009}{
  {\bf Stanford University} \\
   B.S. in Biological Sciences \\
  {\it Advisors:} Elizabeth Hadly \& Rodolfo Dirzo \\
  {\it Honors thesis:} Both neutral and deterministic processes drive
  community structure
}

\section{Appointments}
%
\cventry{2017}{
  {\bf Omidyar Fellow,} Santa Fe Institute
}
%
\cventry{2016}{
  {\bf Postdoctoral Fellow,} Berkeley Initiative in Global Change
  Biology, UC Berkeley
}
%
\cventry{2010}{
  {\bf Fulbright Scholar,} Pontificia Universidad Cat{\'o}lica de Chile
}

\section{Peer-reviewed Publications}
%
\cventry{in press}{
\begin{itemize}
\item[]\hspace{-1.1\leftmargin} O'Dwyer JP, {\bf Rominger AJ}, Xiao X
  (2017). Reinterpreting Maximum Entropy in Ecology: a null hypothesis
  constrained by ecological mechanism. {\it Ecology
    Letters}. Preprint: {\tt  arxiv.org/abs/1702.08543}.
\item[]\hspace{-1.51\leftmargin} Harte J, Newman EA, {\bf Rominger
    AJ}. (2017) Metabolic partition across individuals in ecological
  communities. {\it Global Ecology and Biogeography}. Online early:
  {\tt http://onlinelibrary.wiley.com/doi/10.1111/geb.12621/full}
  \end{itemize}
}
%
\cventry{in review}{
\begin{itemize}
\item[]\hspace{-1.1\leftmargin} {\bf Rominger AJ}, Overcast I,
  Krehenwinkel H, Gillespie RG, Harte L, Hickerson, MJ (in rev.). Linking
  evolutionary and ecological theory illuminates non-equilibrium
  biodiversity. In review at {\it Trends in Ecology and Evolution}.
  Preprint: {\tt arxiv.org/abs/1705.04725}.
\item[]\hspace{-1.1\leftmargin} {\bf Rominger AJ}, Fuentes MA, Marquet
  PA. (in rev.) Punctuated non-equilibrium and niche conservatism
  explain biodiversity fluctuations through the Phanerozoic. In review
  at {\it Nature Ecology Evolution}. Preprint: {\tt
    arxiv.org/abs/1707.09268}.
  \end{itemize}
}
%
\cventry{2017}{
  \begin{itemize}
  \item[]\hspace{-1.1\leftmargin} {\bf Rominger AJ}, Merow C. (2017)
    {\tt meteR}: An {\tt R} package for testing the Maximum Entropy
    Theory of Ecology. {\it Methods in Ecology and Evolution} 8:
    241--247.
  \item[]\hspace{-1.1\leftmargin} Stegner MA, Karp DS, {\bf Rominger
      AJ}, Hadly EA (2017). Can protected areas really maintain mammalian
    diversity? Insights from a nestedness analysis of the Colorado
    Plateau. {\it Biological Conservation} 209: 546--553.
  \end{itemize}
}
%
\cventry{2016}{
  \begin{itemize}
  \item[]\hspace{-1.1\leftmargin} {\bf Rominger AJ}, {\it et
      al.}. (2016). Community assembly on isolated islands: Macroecology meets
    evolution. {\it Global Ecology and Biogeography} 25: 769--780.
    % 
  \item[]\hspace{-1.1\leftmargin} {\bf Rominger AJ} (2016) Ecological Theories in Biogeography. In: Kliman,
    R.M. (ed.), {\it Encyclopedia of Evolutionary Biology} 1:
    145–-148. Oxford: Academic Press.
    % 
  \item[]\hspace{-1.1\leftmargin} Sardinas HS, Tom K, {\bf Rominger
      AJ}, Kremen C. (2016). Patterns of native bee crop pollination within
    agricultural fields are limited by nest site location. In press at
    {\it Ecological Applications} 26: 438--447.
  \end{itemize}
}
%
\cventry{2015}{
  \begin{itemize}
  \item[]\hspace{-1.1\leftmargin} Harte J, {\bf Rominger AJ}, Zhang
    W. (2015). Integrating macroecological metrics and community taxonomic
    structure. {\it Ecology Letters} 18: 1068--1077.
  \end{itemize}
}
%
\cventry{2013}{
  \begin{itemize}
  \item[]\hspace{-1.1\leftmargin} Harte J, Kitzes J, Newman E \& {\bf
      Rominger AJ}. (2013). Taxon categories and the universal
    species-area relationship: A comment on Sizling et al.  {\it The
      American Naturalist} 181: 282--287.
  \end{itemize}
}
%
\cventry{2012}{
  \begin{itemize}%\itemsep0em
  \item[]\hspace{-1.1\leftmargin} Maurer BA, Kembel SW, {\bf Rominger
      AJ} \& McGill BJ. (2012). Estimating metacommunity extent using
    data on species abundances, environmental variation, and
    phylogenetic relationships across geographic space. {\it
      Ecological Informatics} 13: 114--122.
  % 
  \item[]\hspace{-1.1\leftmargin} Karp DS, {\bf Rominger AJ}, Zook J,
    Ranganathan J, Ehrlich PR \& Daily GC. (2012). Intensive
    agriculture erodes $\beta$-diversity
    at large scales. {\it Ecology Letters} 15: 963--970.
  \end{itemize}
}
%
\cventry{2009}{
  \begin{itemize}
  \item[]\hspace{-1.1\leftmargin} {\bf Rominger AJ}, Miller TEX \& Collins
    SL. (2009). Relative contributions of neutral and niche-based
    processes to the structure of a desert grassland grasshopper
    community. {\it Oecologia} 161: 791--800.
  \end{itemize}
}
% %
% \cventry{submitted}{
%   \begin{itemize}
%   \item[]\hspace{-1.1\leftmargin} {\bf Rominger AJ}, Fuentes MA,
%     Marquet PA. (submitted). Punctuated non-equilibrium in the volatility of
%     macroevolution drives complex trajectories of Phanerozoic
%     diversity. Submitted to {\it Nature Ecology and Evolution}.
%   \end{itemize}
% }

\section{Authored Software}
%
\cventry{2016}{
  \begin{itemize}
  \item[]\hspace{-1.1\leftmargin} {\bf Rominger AJ}, {\tt pika}: An
    {\tt R} package for testing and visualization
    macroecology. \url{https://github.com/ajrominger/pika}
  \end{itemize}
}
%
\cventry{2015}{
  \begin{itemize}
  \item[]\hspace{-1.1\leftmargin} {\bf Rominger AJ}, Merow
    C. (2015). {\tt meteR}: Testing the Maximum Entropy Theory of Ecology. R
    package version 1.0.  \url{http://CRAN.R-project.org/package=meteR}
  \end{itemize}
}

\section{Grants}
%
\cventry{2016}{
 {\bf NSF Biocollections Postdoctoral Fellow,} University of
 Florida. \\ 
(declined) \$137,000 \\
  \\[-0.75em]
{\bf NIMBioS Postdoctoral Fellow,} University of Tennessee. \\
(declined) \$108,000
}
%
\cventry{2015}{
 {\bf Philomathia Graduate Student Fellowship,} UC Berkeley. \\ \$20,000
}
\cventry{2014}{
  {\bf Berkeley Initiative in Global Change Biology Workshop Grant,}
  UC Berkeley. \\ \$10,000
}
%
\cventry{2012--2017}{
  {\bf National Science Foundation Grant DEB 1241253:} Dimensions of
  Biodiversity---Community level approach to understanding speciation
  in Hawaiian lineages. I contributed to the design and writing of grant
  sections dealing with sampling strategy, statistical analysis and
  ecological theory testing and development.
}
%
\cventry{2011--2015}{
  {\bf Graduate Research Fellowship,} National Science
  Foundation. \\ \$121,000 \\
  \\[-0.75em]
  {\bf Walker Fund for Entomology,} Essig Museum of
  Entomology. \\ \$4,200 }

\section{Awards \& Honors}
%
\cventry{2015}{
  {\bf Outstanding GSI Award,} University of California, Berkeley \\
  {\bf Usinger Award in Entomology,} University of California, Berkeley
}
%
\cventry{2009}{
  {\bf Kennedy Prize for Outstanding Honors Thesis,} Stanford
  University. Given to one thesis in the Natural Sciences \\
  \\[-0.75em]
  {\bf Firestone Medal for Excellence in Undergraduate Research,} \\
  Stanford University. Given to ten finishing students in the Department of Biology \\
  \\[-0.75em]
  {\bf Award for Excellence in Teaching,} Stanford University
}

\section{Organized workshops}
%
\cventry{2014}{ 
  {\bf Big ecological questions, diverse data, new methods}. I
  organized and secured funding from the Berkeley Initiative in Global
  Change Biology for a workshop bringing together leaders in
  ecological theory, statistics and data digitization efforts to help
  map future directions for ecoinformatics.
}
%
\cventry{2012--2013}{
  {\bf Global change biogeography}. Created and lead a Berkeley Initiative in
  Global Change Biology working group.
}

\section{Invited Talks}
%
\cventry{2016}{
  \begin{itemize} % first item was strangely offset so took off extra 0.1em
  \item[]\hspace{-1.2\leftmargin} {\bf Rominger AJ}. (2016). Isolated
    islands untangle universal patterns at the nexus of macroevolution
    and macroecology. {\it Island Biology 2016}. Terceira Island,
    Azores, Portugal.
\end{itemize}
}
%
\cventry{2015}{
  \begin{itemize} % first item was strangely offset so took off extra 0.1em
  \item[]\hspace{-1.2\leftmargin} {\bf Rominger AJ}. (2015). Community
    assembly on isolated islands: Macroecology meets evolution. {\it
      Evolution 2015}. Sao Palo, Brazil.
\end{itemize}
}
%
\cventry{2014}{
  \begin{itemize} % first item was strangely offset so took off extra 0.1em
  \item[]\hspace{-1.2\leftmargin} {\bf Rominger AJ}. (2014). Theory
    based perspectives on global change biology. Berkeley Initiative
    in Global Change Biology site visit by the Moore Foundation.
\end{itemize}
}
%
\cventry{2013}{
  \begin{itemize} % first item was strangely offset so took off extra 0.1em
  \item[]\hspace{-1.2\leftmargin} {\bf Rominger
      AJ}. (2013). Evolutionary constraints and information entropy in
    ecology. {\it $98^{th}$ Ecological Society of America Annual
      Meeting}. Minneapolis, MN, USA.
\end{itemize}
}
%
\cventry{2012}{
  \begin{itemize} % first item was strangely offset so took off extra 0.1em
  \item[]\hspace{-1.2\leftmargin} {\bf Rominger
      AJ}. (2012). Specimen-based biogeography: Imperfect detection
    and biased sampling. {\it $6^{th}$ Biannual Meeting of the
      International Biogeography Society}. Miami, FL, USA.
\end{itemize}
}
%
\cventry{2011}{
  \begin{itemize} % first item was strangely offset so took off extra 0.1em
  \item[]\hspace{-1.2\leftmargin} {\bf Rominger AJ}, Gruner D, Harte J
    \& Gillespie RG. (2011). Making and breaking a new ecological
    theory. {\it Evolution of the Pacific}. Honolulu, HI, USA.
\end{itemize}
}

\section{Selected Conference Presentations}
%
\cventry{2016}{
  \begin{itemize}
  \item[]\hspace{-1.1\leftmargin} {\bf Rominger AJ}. How to be happy
    when your data are SAD. {\it $101^{st}$ Ecological Society of
      America Annual Meeting}. Ft. Lauderdale, FL, USA.
  \end{itemize}
}
%
\cventry{2015}{ 
  \begin{itemize}
  \item[]\hspace{-1.1\leftmargin} {\bf Rominger AJ}, Gillespie
    R. (2015). Macroevolutionary signals of insular adaptive
    radiations: Synthesizing across island systems with a novel
    statistical method.  {\it $7^{th}$ Biannual Meeting of the
      International Biogeography Society}. Bayreuth, Germany.
  \end{itemize}
}
%
\cventry{2014}{
  \begin{itemize}
  \item[]\hspace{-1.1\leftmargin} {\bf Rominger AJ}, M'Gonigle L,
    Maher SP, Iknayan KJ, Chang L, Rapacciuolo G, Holroyd
    P. (2014). Estimating community change from sporadic data: A novel
    statistical technique sheds light on continental-scale ecology of
    the Pleistocene-Holocene transition. {\it $99^{th}$ Ecological
      Society of America Annual Meeting}. Sacramento, CA, USA.
  \end{itemize}
}
%
\cventry{2012}{
  \begin{itemize}
  \item[]\hspace{-1.1\leftmargin} {\bf Rominger AJ}, Gruner D, Harte J
    \& Gillespie RG. (2012). Making and breaking a new ecological
    theory. {\it $97^{th}$ Ecological Society of America Annual
      Meeting}. Portland, OR, USA.
  % \item[]\hspace{-1.1\leftmargin} {\bf Rominger AJ}. (2012). Spatial
  %   homogenization of North American bird communities through
  %   time. {\it $6^{th}$ Biannual Meeting of the International
  %     Biogeography Society}. Miami, FL, USA.
  \end{itemize}
}
%
\cventry{2011}{
  \begin{itemize}
  \item[]\hspace{-1.1\leftmargin} {\bf Rominger AJ}, Fuentes MA \&
    Marquet PA. (2011). Volatility of clade-specific random walks
    evolves across lineages and drives complex diversification
    patterns through geologic time. {\it $96^{th}$ Ecological Society
      of America Annual Meeting}. Austin, TX, USA.
  \end{itemize}
}
%
\cventry{2009}{
  \begin{itemize}
  \item[]\hspace{-1.1\leftmargin} {\bf Rominger AJ} \& Hadly
    EA. (2009). Geographic diffusion of New World bird species:
    Energetics, inter-continental dispersal, vicariance and
    diversification. {\it $4^{th}$ Biannual Meeting of the
      International Biogeography Society}. Merida, Yucatan, Mexico.
  \end{itemize}
}
% 
\cventry{2007}{
  \begin{itemize}
  \item[]\hspace{-1.1\leftmargin} {\bf Rominger AJ}, Miller TEX \&
    Collins SL. (2007). Dispersal, determinism and the structure of a
    local grasshopper community. {\it $92^{nd}$ Ecological Society of
      America Annual Meeting}. San Jose, CA, USA.
  \end{itemize}
}

\section{Teaching Experience}
%
\cventry{2014}{
  {\bf Graduate Student Instructor,} UC Berkeley \\
  ESPM 174: Design and Analysis of Ecological Studies \\
  {\it Instructor:} Perry de Valpine \\
  {\bf Graduate Student Instructor,} UC Berkeley \\
  INTEGRATIVE BIOLOGY 166: Evolutionary Biogeography \\
  {\it Instructor:} Anthony Barnosky
}
%
\cventry{2010--2013}{
  {\bf R Tutorials,} Stanford University and UC Berkeley \\
  \vspace{-2em}
  \begin{itemize}
  \item[] {\tt R} and phylogenetics. Evo Lab group, UC Berkeley {\it December 2013}  
  \item[] General {\tt R}. Evo Lab group, UC Berkeley {\it December 2012}
  \item[] Advanced {\tt R} plotting. Hadly Lab group, Stanford University {\it March 2010}
  \item[] General {\tt R}. Hadly Lab group, Stanford University {\it December 2011}
  \end{itemize}
}
%
\cventry{2009}{
  {\bf Teaching Assistant,} Stanford University \\
  BIOLOGY 121: Biogeography \\
  {\it Instructor:} Elizabeth Hadly
}

\section{Mentoring Experience}
%
\cventry{current}{
  \begin{itemize}
  \item[]\hspace{-1.1\leftmargin} Karen Gallardo: Soundscape evolution across the
    Hawaiian chronosequence. 
  \item[]\hspace{-1.1\leftmargin} Edward Huang: Scientific computing
    and biocollections database management with {\tt R}.
  \end{itemize}
}
%
\cventry{2016}{
  \begin{itemize}
  \item[]\hspace{-1.1\leftmargin} Brittany Mathat: Phylogeny and
    biogeography of native Hawaiian {\it Nabis}.
  \end{itemize}
}
%
\cventry{2015}{
  \begin{itemize}
  \item[]\hspace{-1.1\leftmargin} Kelley Henning: Microbiome of native Hawaiian
    arthropods
  \item[]\hspace{-1.1\leftmargin} Crispin Herrick: Populations genetics of native
    Hawaiian spiders
  \item[]\hspace{-1.1\leftmargin} Terisa Yiin: Phylogeography, population genomics
    and speciation mechanisms of the spider genus {\it Ariamnes}
  \end{itemize}
}
%
\cventry{2014}{
  \begin{itemize}
  \item[]\hspace{-1.1\leftmargin} Victoria Knorr: Integrating the Red Queen hypothesis
    with biogeography using fossil mammals
\end{itemize}
}
%
\cventry{2013}{
  \begin{itemize}  
  \item[]\hspace{-1.1\leftmargin} Addien Wray: Analysis of island $\beta$-diversity
    patterns and land-use change.
  \end{itemize}
}

\section{Outreach}
%
\cventry{2013--present}{
  {\bf Board Member} \\
  Talking Talons Youth Leadership Community Fund, an organization that
  funds environmental education projects.
}

\cventry{2009--present}{
  {\bf Community presentation speaker} \\
  Present at youth and environmental group meetings including Central
  New Mexico Audubon Society and Pacific Internship Programs for
  Exploring Science about science, conservation and environmental
  education.
}

\cventry{2009}{
  {\bf Splash instructor} \\
  Thought an interactive course about global change biology to K-12
  students as part of Stanford University's Educational Studies Splash Program.
}

\cventry{2006--2012}{
  {\bf Natural history docent} \\
  Lead classroom and community tours of Jasper Ridge Biological
  Preserve (Stanford University) focusing on local
  conservation issues, ecology, evolution and geology.
}

\section{Professional Service}
%
\begin{tabular}[t]{ll}
  Referee for: 
  & {\it Ecology Letters} \\
  & {\it Proceedings of the Royal Society B} \\
  & {\it The American Naturalist} \\
  & {\it Ecology} \\
  & {\it Journal of Theoretical Biology} \\
  & {\it Oecologia} \\
  & {\it PLoS ONE}
\end{tabular}

\end{document}
